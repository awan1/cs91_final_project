\section{Project Proposal}

\subsection{Problem}
\textbf{A description of the problem you plan to solve and the motivations for doing so (i.e., why this problem is interesting/important).}

Twitter is a social media platform with 284 million users active per month, tweeting approximately 500 million tweets
per day.~\cite{twitter} At a maximum of 140 characters per tweet and about 1 byte per character\footnote{Issues arise with different
encodings, but we ignore these for the sake of our back-of-the-envelope calculation} this represents an informational
flow of over 70GB of text per day. An average book is about 2MB in size~\cite{bookfact}, so Twitter users are collectively
writing about 35,000 books a day.

With this wealth of textual information that is often supplemented with geo-location data and content-connecting hashtags,
it is no surprise that a multitude of tools have arisen to harvest the information encoded in tweets. Fabric, Twitter's
API~\cite{twitterAPI}, has


\subsection{Goals}
\textbf{The goals you have for the project. What constitutes success and how will you evaluate it?}

Our goal for this project is to have a fu

\subsection{Hypothesis}
\textbf{Your hypothesis: given the work you intend to do, what are the results you expect to see? How does this work help to solve the problem?}

\subsection{Environment}
\textbf{Characterize environment you intend to operate in. Does your project operate on Amazon Web Services? on the general Internet? in a data center?}

We intend to operate using Amazon Web Services. \sys consists of an always-on EC2 instance